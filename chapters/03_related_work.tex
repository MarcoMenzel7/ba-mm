% !TeX root = ../main.tex
% Add the above to each chapter to make compiling the PDF easier in some editors.

\chapter{Related Work}\label{chapter:related_work}

Digitalization has imposed great changes to the way grocery are purchased.
But while there is an increasing interest in online grocery shopping platforms, there is also progress in digitalizing the shopping experience in local supermarkets.\\

In 2011 Heinrichs et. al. proposed "The Hybrid Shopping List ~\cite{heinrichs2011hybrid}. It featured a combination of a digitalised and a paper-based list. In their work they found that only a very small part of used shopping lists were digital. The hybrid approach tried to make digital shopping lists more accessible, by still allowing the usage of traditional shopping lists. 
Using a digital pen, a paper based shopping list could be appended to the digital list. The digitalised list allready included some of the appealing features of digitalised shopping lists, like synchronisation between multiple devices within a household.

Focusing on the retailer side of technological advances, there are a lot of different proposals for the use of "Smart shopping Carts". One approach \cite{7932080}
uses an inbuild RFID-reader in the cart and RFID-tags in the items to automate billing. Also it can help with the logistics in the store, so if items run low, staff is informed automaticly. Other approaches use indoor navigation in combination with a shopping list to help customers path the store efficiently. Both approaches can improve the shopping experience for the customer. Attacking bottlenecks like searching items, waiting in line at the cash register, or the payment process itself enables customers to reduce their time in the store and the retailers to generate more revenue \cite{inproceedings}.

A very sophisticated way to reduce or extinguish the bottleneck, induced by the payment process, comes from Amazon. In the Amazon Go stores \cite{wankhede2018just} customers are tracked using computervision. After a check-in with their phone, customers are tracked by optical sensors. The sensors detect if a customer picks up an item or put it back. With this information the items of every customer are tracked and the customer is automaticaly charged upon leaving the store.