% !TeX root = ../main.tex
% Add the above to each chapter to make compiling the PDF easier in some editors.

\chapter{Conclusion and future work}\label{chapter:conc_future_work}
Our evaluation shows two things. The preprocessor can solve the problem it was initialy written for. This is cleaning up grocery lists and making them more compact.
On the other hand we see that the cleaned up versions created by the preprocesser loose information for other systems that require context. The system would require multiple preprocessers for the seperate tasks at hand, as their purpose diverges.
On one hand, we want to create an easily readable shopping list, without any irrelevant information. This allows us to purchase with higher efficency, saving us time and possibly money. On the other hand, knowledge on how a given ingredient could be processed yields relevant information about which food category an item belongs to. In further development of  this project a preprocesser more tailored for the categorization-problem would be required. The properties described in \ref{sub:problem} should be implementet into an additional preprocesser.

The unit converter as described in \ref{section:conversion} still requires further development aswell. While we have added some additional systems, the system is still not universally usable. Right now only one of imperial or US-customary units can be processed. This disables us from combining recipies containing units of different systems into one mealplan. To solve this problem we could extend our ingredient model by an additional attribute for the system the unit is in. Furthermore another yet unsolved problem is the combination of different types of units. If our mealplan features recipes that contain volume and mass unit types aswell as counted items, the list will contain three instances of the same item. A quick fix for this would be an augmentation of the grocery list item. If we change the items to contain lists of quantites and units, similar to the json we are using to add recipes, we could reduce the instances to one. The best possible solution for this problem would be an inclusion of a database that features density aswell as the mass of an average item. This would allow us to combine the different unit types.\\

In \ref{sec:eval_list} we have shown that the preprocesser can allready help cleaning up the grocery shopping list. The preprocesser is easily maintainable, by including a lot of configuration items. While adding our testset(\ref{sub_testset}) we have seen that this maintainance is also required. An approach solely based on regular expressions can solve its task well, but will probably require much more maintainance to deliver similar results, compared to a maschine learning approach.

